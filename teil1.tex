\section{Historischer Abriss der Astro(teilchen)physik}

\subsection{Erste Vorlesung}
\begin{itemize}
\item Astrophysik ist der Teil der Wissenschaft, der am weitesten in die Vergangenheit des Menschen zurück reicht\\
      $\longrightarrow$ Beginn etwa 4000 Jahre v.Chr.
\item  Dokumentation durch Höhlenmalereien, Himmelsscheibe, Stonehang\\
      $\longrightarrow$ großes Interesse da überlebenswichtige Bedeutung für die Menschen, z.B. Bestimmung der Jahreszeiten
\item Supernovae gut dokumentiert, da besonders Imposant für Menschen
\item Sonnensystem und Modelle zur Beschreibung der Rolle der Erde im Universum\\
  	 $\longrightarrow$ zuerst konzentrisches Weltbild ()\\
     $\longrightarrow$ später Weltbild (Kopernikus), weitere Erklärung unter anderem mit Gravitation durch Newton\\
     $\longrightarrow$ keine Kreise sondern Ellipsen (Keppler)
\item Messier-Katalog
\item neuere Entdeckungen:

\end{itemize}

\textbf{Wichtigste Erkenntnisse:}\\
$\Longrightarrow$ Astronomie hat lange zurückreichende Geschichte\\
$\Longrightarrow$ besonders Auffallende und Spektakuläre Ereignisse tragen Astronomie auch heute noch vorran

\subsection{Zweite Vorlesung}
\begin{itemize}
\item Erdatmosphere durchlässig für zwei Frequenzbereiche\\
$\longrightarrow$ sichtbares Licht und Radiobereich
\item Infrarot-, Gamma-, Röntgenbereich wird über dem Erdboden absorbiert
$\longrightarrow$ versch. Methoden der Beobachtung von Flugzeug über Ballon bis Satelliten
\item Sonne produziert aus Wasserstoff schwerere Elemente\\
$\longrightarrow$
\item Oberflächentemperatur beträgt $5800\,\text{°C}$
\item Temperatur in Korona steigt auf mehrere $10^6\,\text{°C}$\\
$\longrightarrow$ Energiemäßig möglich da Dichte stark abnimmt\\
$\longrightarrow$ genauer Prozess unverstanden
\item unterschiedlich starkes Nachfließen von heißen ganzen aus dem Inneren führen zu gekräuselter Oberfläche
\item Stellen an denen starke Magnetfelder austreten sind relativ zur Umgebung kälter
$\longrightarrow$ Sonnenflecken (etwa $4000\,\text{°C}$)
\item Sonne macht $99,2\,\%$ der Masse unseres Sonnensystems aus
\item Erkenntnisse über Sonne durch betrachtung in verschiedenen Frequenzbereichen
\item Erde größter bekannter massiver Planet in unserem System (in anderen Systemen auch Supererden mit 3-4 Erdmassen)
\item Abstand Sonne Erde ist 1AE ($150\cdot 10^6\,\text{m}$)
\item äußere Planeten sind Gasriesen\\
$\longrightarrow$ niedriegere Dichten aber viel größere Volumen
\item Pluto hat Abstand 40AE zur Sonne
\item Hinweise auf weitere Planeten da Asteriodenbahnen durch Gravitation unentdeckter Massen beeinflusst werden
\item Universium sehr leer\\
$\longrightarrow$ an besonders dichten Stellen 100 Teilchen pro $\text{cm}^3$
\item Sonnen entstehen wenn an dichten Stellen Teilchen aneinander gepresst werden, z.B durch Supernovae, Sonnenexplosionen
\item erst Deterium-Brennen da energetisch bevorzugt\\
$\longrightarrow$ anschließend wird bei fast allen Sternen Heliumbrennen aktiviert (Ausnahme: Braune Zwerge)
\item unsere Sonne ist ein Hauptreihenstern
\item Versuche von Spektrallinien auf Temperatur zu schließen sind Fehlgeschlagen, da Annahme, dass Sonne materialtechnisch wie Erde aufgebaut ist, falsch war

\end{itemize}

\textbf{Wichtigste Erkenntnisse:}\\
$\Longrightarrow$ Zugang zu Objekten im Weltall über Strahlung\\
$\Longrightarrow$ Übersicht über Planten in unserem Sonnensystem (noch ergänzen)\\
$\Longrightarrow$ Klassifizieren von Sonnen
