\section{Historischer Abriss der Astro(teilchen)physik}

\subsection{Erste Vorlesung}
\begin{itemize}
\item Astrophysik ist der Teil der Wissenschaft, der am weitesten in die Vergangenheit des Menschen zurück reicht\\
      $\longrightarrow$ Beginn etwa 4000 Jahre v.Chr.
\item  Dokumentation durch Höhlenmalereien, Himmelsscheibe, Stonehang\\
      $\longrightarrow$ großes Interesse da überlebenswichtige Bedeutung für die Menschen, z.B. Bestimmung der Jahreszeiten
\item Supernovae gut dokumentiert, da besonders Imposant für Menschen
\item Sonnensystem und Modelle zur Beschreibung der Rolle der Erde im Universum\\
  	 $\longrightarrow$ zuerst konzentrisches Weltbild ()\\
     $\longrightarrow$ später Weltbild (Kopernikus), weitere Erklärung unter anderem mit Gravitation durch Newton\\
     $\longrightarrow$ keine Kreise sondern Ellipsen (Keppler)
\item Messier-Katalog
\item neuere Entdeckungen:

\end{itemize}

\textbf{Wichtigste Erkenntnisse:}\\
$\Longrightarrow$ Astronomie hat lange zurückreichende Geschichte\\
$\Longrightarrow$ besonders Auffallende und Spektakuläre Ereignisse tragen Astronomie auch heute noch vorran

\subsection{Zweite Vorlesung}
